\documentclass[a4j,11pt]{jarticle}

% ToDo: 提出要領に従って,適切な余白を設定する
\usepackage[top=25truemm,  bottom=30truemm,
            left=25truemm, right=25truemm]{geometry}


% ToDo: 提出要領に従って,適切なタイトル・サブタイトルを設定する
\title{応用解析 \\
       レポート}

% ToDo: 自分自身の氏名と学生番号に書き換える
\author{氏名: 今田将也 (IMADA, Masaya) \\
        学生番号: 09430509}

% ToDo: 教員の指示に従って適切に書き換える
\date{出題日: 2019年08月29日 \\
      提出日: 2019年08月05日 \\
      締切日: 2019年08月08日 \\}  % 注:最後の\\は不要に見えるが必要.

% ToDo: 図を入れる場合,以下の1行を有効にする
\usepackage{graphicx}
\usepackage{listings}
\lstset{%
  language={C},
  basicstyle={\ttfamily\footnotesize},%
  frame={tb},
  breaklines=true,
  columns=[l]{fullflexible},%
  numbers=left,%
  xrightmargin=0zw,%
  xleftmargin=3zw,%
  numberstyle={\scriptsize},%
  stepnumber=1,
  numbersep=1zw,%
  lineskip=-0.5ex,%
  showstringspaces=false
}

\begin{document}
\maketitle

%%%%%%%%%%%%%%%%%%%%%%%%%%%%%%%%%%%%%%%%%%%%%%%%%%%%%%%%%%%%%%%%
\section{概要}\label{sec:gaiyou}
%%%%%%%%%%%%%%%%%%%%%%%%%%%%%%%%%%%%%%%%%%%%%%%%%%%%%%%%%%%%%%%%
応用解析レポート課題として,
\begin{displaymath}
    \int^{2}_{1} \left(\frac{1}{x^{2}}+\frac{3}{x^{6}}\right) dx
\end{displaymath}
を台形積分およびシンプソン積分で計算をC言語にて行わせた.

本レポートでは,その計算結果と考察をまとめた.また,
ロンバーグ積分による計算も行い表にまとめた.
%%%%%%%%%%%%%%%%%%%%%%%%%%%%%%%%%%%%%%%%%%%%%%%%%%%%%%%%%%%%%%%%
\section{プログラムの使用法}\label{sec:use}
%%%%%%%%%%%%%%%%%%%%%%%%%%%%%%%%%%%%%%%%%%%%%%%%%%%%%%%%%%%%%%%%

本プログラムは台形積分とシンプソン積分を\verb|n=2,4,8,16|の四通りの
分割数で計算するためのプログラムである.
処理結果を標準出力に出力する.
プログラムは第\ref{sec:program}節に添付している.

%%%%%%%%%%%%%%%%%%%%%%%%%%%%%%%%%%%%%%%%%%%%%%%%%%%%%%%%%%%%%%%%
\section{結果}
%%%%%%%%%%%%%%%%%%%%%%%%%%%%%%%%%%%%%%%%%%%%%%%%%%%%%%%%%%%%%%%%

まず,プログラムを実行した台形積分とシンプソン積分の結果を以下に示す.
\begin{verbatim}
    /*daikei*/
    n=2
    S1= 1.42812822
    |S1-S|= 0.34687822
    
    n=4
    S1= 1.17841633
    |S1-S|= 0.09716633
    
    n=8
    S1= 1.10644500
    |S1-S|= 0.02519500
    
    n=16
    S1= 1.08761162
    |S1-S|= 0.00636162
    
    /*sympthon*/
    n=2
    S1= 1.18802512
    |S1-S|= 0.10677512
    
    n=4
    S1= 1.09517903
    |S1-S|= 0.01392903
    
    n=8
    S1= 1.08245456
    |S1-S|= 0.00120456
    
    n=16
    S1= 1.08133383
    |S1-S|= 0.00008383
\end{verbatim}

なお,両方の数値積分の絶対誤差を表\ref{table1}にまとめた.
%ここテーブル%
\begin{table}[tbp]
    \begin{center}
    \begin{tabular}{|l||l|l||l|l|}
    \hline
    n  & 台形積分       & 誤差         & シンプソン積分    & 誤差         \\ \hline
    2  & 1.42812822 & 0.34687822 & 1.18802512 & 0.10677512 \\ \hline
    4  & 1.17841633 & 0.09716633 & 1.09517903 & 0.01392903 \\ \hline
    8  & 1.10644500 & 0.02519500 & 1.08245456 & 0.00120456 \\ \hline
    16 & 1.08761162 & 0.00636162 & 1.08133383 & 0.00008383 \\ \hline
    \end{tabular}
\end{center}
    \caption{台形積分とシンプソン積分の結果と誤差の表}
    \label{table1}
\end{table}
%%%%%%%%%%%%%%%%%%%%%%%%%%%%%%%%%%%%%%%%%%%%%%%%%%%%%%%%%%%%%%%%
\section{考察}
%%%%%%%%%%%%%%%%%%%%%%%%%%%%%%%%%%%%%%%%%%%%%%%%%%%%%%%%%%%%%%%%

ここでは,概要で挙げた以下の項目について考察を述べる.

\begin{enumerate}
\setlength{\parskip}{2pt} \setlength{\itemsep}{2pt}
    \item 台形積分についての考察
    \item シンプソン積分についての考察
\end{enumerate}

%--------------------------------------------------------------%
\subsection{台形積分についての考察}
%--------------------------------------------------------------%
台形積分の結果を見ると\verb|S1-S|の絶対誤差は,\verb|n=2|と\verb|n=4|を見ると
約$0.25$誤差が小さくなっている.また,\verb|n=4|と\verb|n=8|を見ると約$0.07$だけ小さくなっており,
\verb|n=8|と\verb|n=16|を見ると約$0.02$だけ小さくなっており,真の値に近づいていることが
わかり,次第にその絶対誤差が小さくなっている.

また,相対誤差は\verb|n=2|で約$0.32$そして,\verb|n=16|で約$0.005$と減っているいることもわかった.

\verb|n=2|と\verb|n=4|では後者のほうが約$0.28$倍精度が良くなっており,\verb|n=4|と\verb|n=8|でも,約$0.26$倍精度が
良くなっている.\verb|n=8|と\verb|n=16|においては,約$0.25$倍精度が

従って,台形積分は約$\frac{1}{4}$倍ずつ誤差が少なくなると言えるだろう.

%--------------------------------------------------------------%
\subsection{シンプソン積分についての考察}
%--------------------------------------------------------------%
シンプソン積分の結果を見ると\verb|S1-S|の絶対誤差は,\verb|n=2|と\verb|n=4|を見るとその誤差は
約$0.092$小さくなっている.また,\verb|n=4|と\verb|n=8|を見ると約$0.012$だけ小さくなっており,
\verb|n=8|と\verb|n=16|を見ると約$0.001$だけ小さくなっており,真の値に近づいていることが
わかり,次第にその絶対誤差が台形積分よりはるかに小さくなっている.

また,相対誤差は\verb|n=2|で約$0.098$そして,\verb|n=16|で約$0.000077$と大きく減っているいることもわかった.

\verb|n=2|と\verb|n=4|では後者のほうが約$0.1304$倍精度が良くなっており,\verb|n=4|と\verb|n=8|でも,約$0.0864$倍精度が
良くなっている.\verb|n=4|と\verb|n=8|でも,約$0.0695$倍精度が

従って,シンプソン積分は次第に約$\frac{1}{16}$倍ずつ誤差が少なくなっていくと考える.

%%%%%%%%%%%%%%%%%%%%%%%%%%%%%%%%%%%%%%%%%%%%%%%%%%%%%%%%%%%%%%%%
\section{ロンバーグ積分}
%%%%%%%%%%%%%%%%%%%%%%%%%%%%%%%%%%%%%%%%%%%%%%%%%%%%%%%%%%%%%%%%
課題の式をロンバーグ積分にて,$n_{max}$まで行った結果も示す.
\scriptsize
\begin{verbatim}
/*Romberg*/
I1( 1)= 2.14843750
I1( 2)= 1.42812822 I2( 2)= 1.18802512
I1( 4)= 1.17841633 I2( 4)= 1.09517903 I3( 4)= 1.08898930
I1( 8)= 1.10644500 I2( 8)= 1.08245456 I3( 8)= 1.08160626 I4( 8)= 1.08148907
I1(16)= 1.08761162 I2(16)= 1.08133383 I3(16)= 1.08125912 I4(16)= 1.08125361 I5(16)= 1.08125268
\end{verbatim}
\normalsize
なお,結果を表\ref{table2}にまとめた

\begin{table}[tbp]
    \begin{center}
    \begin{tabular}{|l|l|l|l|l|l|}
    \hline
    n  & $I_1(n)$    & $I_2(n)$    & $I_3(n)$    & $I_4(n)$    & $I_5(n)$    \\ \hline
    1  & 2.14843750 & -          & -          & -          & -          \\ \hline
    2  & 1.42812822 & 1.18802512 & -          & -          & -          \\ \hline
    4  & 1.17841633 & 1.09517903 & 1.08898930 & -          & -          \\ \hline
    8  & 1.10644500 & 1.08245456 & 1.08160626 & 1.08148907 & -          \\ \hline
    16 & 1.08761162 & 1.08133383 & 1.08125912 & 1.08125361 & 1.08125268 \\ \hline
    \end{tabular}
    \end{center}
    \caption{ロンバーグ積分の結果の表}
    \label{table2}
\end{table}
%%%%%%%%%%%%%%%%%%%%%%%%%%%%%%%%%%%%%%%%%%%%%%%%%%%%%%%%%%%%%%%%
\section{作成したプログラム}
\label{sec:program}
%%%%%%%%%%%%%%%%%%%%%%%%%%%%%%%%%%%%%%%%%%%%%%%%%%%%%%%%%%%%%%%%
\normalsize
作成したプログラムを以下に添付する.

\begin{lstlisting}
#include<stdio.h>
#include<math.h>
#include<stdlib.h>

#define S 1.08125000
#define a 1
#define b 2

double f(double x){
    double y;
    y=(1/x/x)+(3/x/x/x/x/x/x);
    return y;
}

void daikei(){
    int i,n,m;
    double S1,xi,yi,y0,yn,h;

    for(m=2;m<=16;m*=2){
        n=m;
        y0=f(a);
        yn=f(b);
        h=fabs(b-a)/n;
        S1=(y0+yn)/2.0;//最初と最後の項を計算しとく

        for(i=1;i<n;i++){//残りの項を足していく
            xi=a+h*i;//1つ隣へ
            yi=f(xi);
            S1+=yi;
        }
        S1=S1*h;
        printf("n=%d\nS1=%11.8f\n",m,S1);
        printf("|S1-S|=%11.8f\n\n",fabs(S1-S));

    }
}
void sympthon(){
    int i,n,m;
    double S1,xi,yi,y0,yn,h;

    for(m=2;m<=16;m*=2){
        n=m;
        y0=f(a);
        yn=f(b);
        h=fabs(b-a)/n;
        S1=(y0+yn);

        for(i=1;i<n/2;i++){//繰り返しはn/2まで
            xi=a+h*2*i;//二つ隣まで
            yi=4*f(xi-h)+2*f(xi);//公式4*F_1+2*F_2+4*F_3+...+2*F_n-1
            S1=S1+yi;
        }
        S1=S1+4*f(b-h);//4*F_n-1項目
        S1=h*S1/3;
        printf("n=%d\nS1=%11.8f\n",m,S1);
        printf("|S1-S|=%11.8f\n\n",fabs(S1-S));
    }
}
void Romberg(){
    //横に一段ずつ,上の段を使って計算していく
    //I(2),I(4)を先に計算しているわけではなさそう?
    int n,k,i,j,g;
    double h,T1,s,m,x,t;

    double T[10][10];

    n=1; 
    h=b-a;
    T[1][1]=h*(f(a)+f(b))/2;//I(1);
    printf("I1(%2d)=%11.8f\n",n,T[1][1]);

    for(k=1;k<=log2(16.0);k++){
        s=0;
        //I(1)を使ってノート6.2
        for(i=1;i<=n;i++){
            x=a+(i-0.5)*h;
            s=s+f(x);
        }
        s=(T[k][1]+h*s)/2;
        /*
        I(n/2)を使ってI(n)を出す?
        ノート6.2(k+1段目の初期値?)
        I(2)の値のこと?
        */

        h=h/2;//幅半分
        n=n*2;//分割数2倍

        m=1;
        for(j=1;j<=k;j++){/*漸化式のところ */
            t=T[k][j];//I(n/2)と//初回はI(1)
            T[k+1][j]=s;//I(n)を使って//初回はI(2)に値が入る
            m=m*4;
            s=(m*s-t)/(m-1);//I1(2)が出た
        }
        T[k+1][j]=s;//段の最後に結果を代入

        for(g=1;g<=k+1;g++){
            printf("I%d(%2d)=%11.8f ",g,n,T[k+1][g]);
        }
        printf("\n");
    }
    return;
}

int main(int argc, char *argv[]){
    printf("/*daikei*/\n");
    daikei();
    printf("/*sympthon*/\n");
    sympthon();
    printf("/*Romberg*/\n");
    Romberg();
    return 0;
}
\end{lstlisting}

\end{document}
